
% Default to the notebook output style

    


% Inherit from the specified cell style.




    
\documentclass[11pt]{article}

    
    
    \usepackage[T1]{fontenc}
    % Nicer default font (+ math font) than Computer Modern for most use cases
    \usepackage{mathpazo}

    % Basic figure setup, for now with no caption control since it's done
    % automatically by Pandoc (which extracts ![](path) syntax from Markdown).
    \usepackage{graphicx}
    % We will generate all images so they have a width \maxwidth. This means
    % that they will get their normal width if they fit onto the page, but
    % are scaled down if they would overflow the margins.
    \makeatletter
    \def\maxwidth{\ifdim\Gin@nat@width>\linewidth\linewidth
    \else\Gin@nat@width\fi}
    \makeatother
    \let\Oldincludegraphics\includegraphics
    % Set max figure width to be 80% of text width, for now hardcoded.
    \renewcommand{\includegraphics}[1]{\Oldincludegraphics[width=.8\maxwidth]{#1}}
    % Ensure that by default, figures have no caption (until we provide a
    % proper Figure object with a Caption API and a way to capture that
    % in the conversion process - todo).
    \usepackage{caption}
    \DeclareCaptionLabelFormat{nolabel}{}
    \captionsetup{labelformat=nolabel}

    \usepackage{adjustbox} % Used to constrain images to a maximum size 
    \usepackage{xcolor} % Allow colors to be defined
    \usepackage{enumerate} % Needed for markdown enumerations to work
    \usepackage{geometry} % Used to adjust the document margins
    \usepackage{amsmath} % Equations
    \usepackage{amssymb} % Equations
    \usepackage{textcomp} % defines textquotesingle
    % Hack from http://tex.stackexchange.com/a/47451/13684:
    \AtBeginDocument{%
        \def\PYZsq{\textquotesingle}% Upright quotes in Pygmentized code
    }
    \usepackage{upquote} % Upright quotes for verbatim code
    \usepackage{eurosym} % defines \euro
    \usepackage[mathletters]{ucs} % Extended unicode (utf-8) support
    \usepackage[utf8x]{inputenc} % Allow utf-8 characters in the tex document
    \usepackage{fancyvrb} % verbatim replacement that allows latex
    \usepackage{grffile} % extends the file name processing of package graphics 
                         % to support a larger range 
    % The hyperref package gives us a pdf with properly built
    % internal navigation ('pdf bookmarks' for the table of contents,
    % internal cross-reference links, web links for URLs, etc.)
    \usepackage{hyperref}
    \usepackage{longtable} % longtable support required by pandoc >1.10
    \usepackage{booktabs}  % table support for pandoc > 1.12.2
    \usepackage[inline]{enumitem} % IRkernel/repr support (it uses the enumerate* environment)
    \usepackage[normalem]{ulem} % ulem is needed to support strikethroughs (\sout)
                                % normalem makes italics be italics, not underlines
    

    
    
    % Colors for the hyperref package
    \definecolor{urlcolor}{rgb}{0,.145,.698}
    \definecolor{linkcolor}{rgb}{.71,0.21,0.01}
    \definecolor{citecolor}{rgb}{.12,.54,.11}

    % ANSI colors
    \definecolor{ansi-black}{HTML}{3E424D}
    \definecolor{ansi-black-intense}{HTML}{282C36}
    \definecolor{ansi-red}{HTML}{E75C58}
    \definecolor{ansi-red-intense}{HTML}{B22B31}
    \definecolor{ansi-green}{HTML}{00A250}
    \definecolor{ansi-green-intense}{HTML}{007427}
    \definecolor{ansi-yellow}{HTML}{DDB62B}
    \definecolor{ansi-yellow-intense}{HTML}{B27D12}
    \definecolor{ansi-blue}{HTML}{208FFB}
    \definecolor{ansi-blue-intense}{HTML}{0065CA}
    \definecolor{ansi-magenta}{HTML}{D160C4}
    \definecolor{ansi-magenta-intense}{HTML}{A03196}
    \definecolor{ansi-cyan}{HTML}{60C6C8}
    \definecolor{ansi-cyan-intense}{HTML}{258F8F}
    \definecolor{ansi-white}{HTML}{C5C1B4}
    \definecolor{ansi-white-intense}{HTML}{A1A6B2}

    % commands and environments needed by pandoc snippets
    % extracted from the output of `pandoc -s`
    \providecommand{\tightlist}{%
      \setlength{\itemsep}{0pt}\setlength{\parskip}{0pt}}
    \DefineVerbatimEnvironment{Highlighting}{Verbatim}{commandchars=\\\{\}}
    % Add ',fontsize=\small' for more characters per line
    \newenvironment{Shaded}{}{}
    \newcommand{\KeywordTok}[1]{\textcolor[rgb]{0.00,0.44,0.13}{\textbf{{#1}}}}
    \newcommand{\DataTypeTok}[1]{\textcolor[rgb]{0.56,0.13,0.00}{{#1}}}
    \newcommand{\DecValTok}[1]{\textcolor[rgb]{0.25,0.63,0.44}{{#1}}}
    \newcommand{\BaseNTok}[1]{\textcolor[rgb]{0.25,0.63,0.44}{{#1}}}
    \newcommand{\FloatTok}[1]{\textcolor[rgb]{0.25,0.63,0.44}{{#1}}}
    \newcommand{\CharTok}[1]{\textcolor[rgb]{0.25,0.44,0.63}{{#1}}}
    \newcommand{\StringTok}[1]{\textcolor[rgb]{0.25,0.44,0.63}{{#1}}}
    \newcommand{\CommentTok}[1]{\textcolor[rgb]{0.38,0.63,0.69}{\textit{{#1}}}}
    \newcommand{\OtherTok}[1]{\textcolor[rgb]{0.00,0.44,0.13}{{#1}}}
    \newcommand{\AlertTok}[1]{\textcolor[rgb]{1.00,0.00,0.00}{\textbf{{#1}}}}
    \newcommand{\FunctionTok}[1]{\textcolor[rgb]{0.02,0.16,0.49}{{#1}}}
    \newcommand{\RegionMarkerTok}[1]{{#1}}
    \newcommand{\ErrorTok}[1]{\textcolor[rgb]{1.00,0.00,0.00}{\textbf{{#1}}}}
    \newcommand{\NormalTok}[1]{{#1}}
    
    % Additional commands for more recent versions of Pandoc
    \newcommand{\ConstantTok}[1]{\textcolor[rgb]{0.53,0.00,0.00}{{#1}}}
    \newcommand{\SpecialCharTok}[1]{\textcolor[rgb]{0.25,0.44,0.63}{{#1}}}
    \newcommand{\VerbatimStringTok}[1]{\textcolor[rgb]{0.25,0.44,0.63}{{#1}}}
    \newcommand{\SpecialStringTok}[1]{\textcolor[rgb]{0.73,0.40,0.53}{{#1}}}
    \newcommand{\ImportTok}[1]{{#1}}
    \newcommand{\DocumentationTok}[1]{\textcolor[rgb]{0.73,0.13,0.13}{\textit{{#1}}}}
    \newcommand{\AnnotationTok}[1]{\textcolor[rgb]{0.38,0.63,0.69}{\textbf{\textit{{#1}}}}}
    \newcommand{\CommentVarTok}[1]{\textcolor[rgb]{0.38,0.63,0.69}{\textbf{\textit{{#1}}}}}
    \newcommand{\VariableTok}[1]{\textcolor[rgb]{0.10,0.09,0.49}{{#1}}}
    \newcommand{\ControlFlowTok}[1]{\textcolor[rgb]{0.00,0.44,0.13}{\textbf{{#1}}}}
    \newcommand{\OperatorTok}[1]{\textcolor[rgb]{0.40,0.40,0.40}{{#1}}}
    \newcommand{\BuiltInTok}[1]{{#1}}
    \newcommand{\ExtensionTok}[1]{{#1}}
    \newcommand{\PreprocessorTok}[1]{\textcolor[rgb]{0.74,0.48,0.00}{{#1}}}
    \newcommand{\AttributeTok}[1]{\textcolor[rgb]{0.49,0.56,0.16}{{#1}}}
    \newcommand{\InformationTok}[1]{\textcolor[rgb]{0.38,0.63,0.69}{\textbf{\textit{{#1}}}}}
    \newcommand{\WarningTok}[1]{\textcolor[rgb]{0.38,0.63,0.69}{\textbf{\textit{{#1}}}}}
    
    
    % Define a nice break command that doesn't care if a line doesn't already
    % exist.
    \def\br{\hspace*{\fill} \\* }
    % Math Jax compatability definitions
    \def\gt{>}
    \def\lt{<}
    % Document parameters
    \title{DetritalDoubleDate\_Exploration}
    
    
    

    % Pygments definitions
    
\makeatletter
\def\PY@reset{\let\PY@it=\relax \let\PY@bf=\relax%
    \let\PY@ul=\relax \let\PY@tc=\relax%
    \let\PY@bc=\relax \let\PY@ff=\relax}
\def\PY@tok#1{\csname PY@tok@#1\endcsname}
\def\PY@toks#1+{\ifx\relax#1\empty\else%
    \PY@tok{#1}\expandafter\PY@toks\fi}
\def\PY@do#1{\PY@bc{\PY@tc{\PY@ul{%
    \PY@it{\PY@bf{\PY@ff{#1}}}}}}}
\def\PY#1#2{\PY@reset\PY@toks#1+\relax+\PY@do{#2}}

\expandafter\def\csname PY@tok@w\endcsname{\def\PY@tc##1{\textcolor[rgb]{0.73,0.73,0.73}{##1}}}
\expandafter\def\csname PY@tok@c\endcsname{\let\PY@it=\textit\def\PY@tc##1{\textcolor[rgb]{0.25,0.50,0.50}{##1}}}
\expandafter\def\csname PY@tok@cp\endcsname{\def\PY@tc##1{\textcolor[rgb]{0.74,0.48,0.00}{##1}}}
\expandafter\def\csname PY@tok@k\endcsname{\let\PY@bf=\textbf\def\PY@tc##1{\textcolor[rgb]{0.00,0.50,0.00}{##1}}}
\expandafter\def\csname PY@tok@kp\endcsname{\def\PY@tc##1{\textcolor[rgb]{0.00,0.50,0.00}{##1}}}
\expandafter\def\csname PY@tok@kt\endcsname{\def\PY@tc##1{\textcolor[rgb]{0.69,0.00,0.25}{##1}}}
\expandafter\def\csname PY@tok@o\endcsname{\def\PY@tc##1{\textcolor[rgb]{0.40,0.40,0.40}{##1}}}
\expandafter\def\csname PY@tok@ow\endcsname{\let\PY@bf=\textbf\def\PY@tc##1{\textcolor[rgb]{0.67,0.13,1.00}{##1}}}
\expandafter\def\csname PY@tok@nb\endcsname{\def\PY@tc##1{\textcolor[rgb]{0.00,0.50,0.00}{##1}}}
\expandafter\def\csname PY@tok@nf\endcsname{\def\PY@tc##1{\textcolor[rgb]{0.00,0.00,1.00}{##1}}}
\expandafter\def\csname PY@tok@nc\endcsname{\let\PY@bf=\textbf\def\PY@tc##1{\textcolor[rgb]{0.00,0.00,1.00}{##1}}}
\expandafter\def\csname PY@tok@nn\endcsname{\let\PY@bf=\textbf\def\PY@tc##1{\textcolor[rgb]{0.00,0.00,1.00}{##1}}}
\expandafter\def\csname PY@tok@ne\endcsname{\let\PY@bf=\textbf\def\PY@tc##1{\textcolor[rgb]{0.82,0.25,0.23}{##1}}}
\expandafter\def\csname PY@tok@nv\endcsname{\def\PY@tc##1{\textcolor[rgb]{0.10,0.09,0.49}{##1}}}
\expandafter\def\csname PY@tok@no\endcsname{\def\PY@tc##1{\textcolor[rgb]{0.53,0.00,0.00}{##1}}}
\expandafter\def\csname PY@tok@nl\endcsname{\def\PY@tc##1{\textcolor[rgb]{0.63,0.63,0.00}{##1}}}
\expandafter\def\csname PY@tok@ni\endcsname{\let\PY@bf=\textbf\def\PY@tc##1{\textcolor[rgb]{0.60,0.60,0.60}{##1}}}
\expandafter\def\csname PY@tok@na\endcsname{\def\PY@tc##1{\textcolor[rgb]{0.49,0.56,0.16}{##1}}}
\expandafter\def\csname PY@tok@nt\endcsname{\let\PY@bf=\textbf\def\PY@tc##1{\textcolor[rgb]{0.00,0.50,0.00}{##1}}}
\expandafter\def\csname PY@tok@nd\endcsname{\def\PY@tc##1{\textcolor[rgb]{0.67,0.13,1.00}{##1}}}
\expandafter\def\csname PY@tok@s\endcsname{\def\PY@tc##1{\textcolor[rgb]{0.73,0.13,0.13}{##1}}}
\expandafter\def\csname PY@tok@sd\endcsname{\let\PY@it=\textit\def\PY@tc##1{\textcolor[rgb]{0.73,0.13,0.13}{##1}}}
\expandafter\def\csname PY@tok@si\endcsname{\let\PY@bf=\textbf\def\PY@tc##1{\textcolor[rgb]{0.73,0.40,0.53}{##1}}}
\expandafter\def\csname PY@tok@se\endcsname{\let\PY@bf=\textbf\def\PY@tc##1{\textcolor[rgb]{0.73,0.40,0.13}{##1}}}
\expandafter\def\csname PY@tok@sr\endcsname{\def\PY@tc##1{\textcolor[rgb]{0.73,0.40,0.53}{##1}}}
\expandafter\def\csname PY@tok@ss\endcsname{\def\PY@tc##1{\textcolor[rgb]{0.10,0.09,0.49}{##1}}}
\expandafter\def\csname PY@tok@sx\endcsname{\def\PY@tc##1{\textcolor[rgb]{0.00,0.50,0.00}{##1}}}
\expandafter\def\csname PY@tok@m\endcsname{\def\PY@tc##1{\textcolor[rgb]{0.40,0.40,0.40}{##1}}}
\expandafter\def\csname PY@tok@gh\endcsname{\let\PY@bf=\textbf\def\PY@tc##1{\textcolor[rgb]{0.00,0.00,0.50}{##1}}}
\expandafter\def\csname PY@tok@gu\endcsname{\let\PY@bf=\textbf\def\PY@tc##1{\textcolor[rgb]{0.50,0.00,0.50}{##1}}}
\expandafter\def\csname PY@tok@gd\endcsname{\def\PY@tc##1{\textcolor[rgb]{0.63,0.00,0.00}{##1}}}
\expandafter\def\csname PY@tok@gi\endcsname{\def\PY@tc##1{\textcolor[rgb]{0.00,0.63,0.00}{##1}}}
\expandafter\def\csname PY@tok@gr\endcsname{\def\PY@tc##1{\textcolor[rgb]{1.00,0.00,0.00}{##1}}}
\expandafter\def\csname PY@tok@ge\endcsname{\let\PY@it=\textit}
\expandafter\def\csname PY@tok@gs\endcsname{\let\PY@bf=\textbf}
\expandafter\def\csname PY@tok@gp\endcsname{\let\PY@bf=\textbf\def\PY@tc##1{\textcolor[rgb]{0.00,0.00,0.50}{##1}}}
\expandafter\def\csname PY@tok@go\endcsname{\def\PY@tc##1{\textcolor[rgb]{0.53,0.53,0.53}{##1}}}
\expandafter\def\csname PY@tok@gt\endcsname{\def\PY@tc##1{\textcolor[rgb]{0.00,0.27,0.87}{##1}}}
\expandafter\def\csname PY@tok@err\endcsname{\def\PY@bc##1{\setlength{\fboxsep}{0pt}\fcolorbox[rgb]{1.00,0.00,0.00}{1,1,1}{\strut ##1}}}
\expandafter\def\csname PY@tok@kc\endcsname{\let\PY@bf=\textbf\def\PY@tc##1{\textcolor[rgb]{0.00,0.50,0.00}{##1}}}
\expandafter\def\csname PY@tok@kd\endcsname{\let\PY@bf=\textbf\def\PY@tc##1{\textcolor[rgb]{0.00,0.50,0.00}{##1}}}
\expandafter\def\csname PY@tok@kn\endcsname{\let\PY@bf=\textbf\def\PY@tc##1{\textcolor[rgb]{0.00,0.50,0.00}{##1}}}
\expandafter\def\csname PY@tok@kr\endcsname{\let\PY@bf=\textbf\def\PY@tc##1{\textcolor[rgb]{0.00,0.50,0.00}{##1}}}
\expandafter\def\csname PY@tok@bp\endcsname{\def\PY@tc##1{\textcolor[rgb]{0.00,0.50,0.00}{##1}}}
\expandafter\def\csname PY@tok@fm\endcsname{\def\PY@tc##1{\textcolor[rgb]{0.00,0.00,1.00}{##1}}}
\expandafter\def\csname PY@tok@vc\endcsname{\def\PY@tc##1{\textcolor[rgb]{0.10,0.09,0.49}{##1}}}
\expandafter\def\csname PY@tok@vg\endcsname{\def\PY@tc##1{\textcolor[rgb]{0.10,0.09,0.49}{##1}}}
\expandafter\def\csname PY@tok@vi\endcsname{\def\PY@tc##1{\textcolor[rgb]{0.10,0.09,0.49}{##1}}}
\expandafter\def\csname PY@tok@vm\endcsname{\def\PY@tc##1{\textcolor[rgb]{0.10,0.09,0.49}{##1}}}
\expandafter\def\csname PY@tok@sa\endcsname{\def\PY@tc##1{\textcolor[rgb]{0.73,0.13,0.13}{##1}}}
\expandafter\def\csname PY@tok@sb\endcsname{\def\PY@tc##1{\textcolor[rgb]{0.73,0.13,0.13}{##1}}}
\expandafter\def\csname PY@tok@sc\endcsname{\def\PY@tc##1{\textcolor[rgb]{0.73,0.13,0.13}{##1}}}
\expandafter\def\csname PY@tok@dl\endcsname{\def\PY@tc##1{\textcolor[rgb]{0.73,0.13,0.13}{##1}}}
\expandafter\def\csname PY@tok@s2\endcsname{\def\PY@tc##1{\textcolor[rgb]{0.73,0.13,0.13}{##1}}}
\expandafter\def\csname PY@tok@sh\endcsname{\def\PY@tc##1{\textcolor[rgb]{0.73,0.13,0.13}{##1}}}
\expandafter\def\csname PY@tok@s1\endcsname{\def\PY@tc##1{\textcolor[rgb]{0.73,0.13,0.13}{##1}}}
\expandafter\def\csname PY@tok@mb\endcsname{\def\PY@tc##1{\textcolor[rgb]{0.40,0.40,0.40}{##1}}}
\expandafter\def\csname PY@tok@mf\endcsname{\def\PY@tc##1{\textcolor[rgb]{0.40,0.40,0.40}{##1}}}
\expandafter\def\csname PY@tok@mh\endcsname{\def\PY@tc##1{\textcolor[rgb]{0.40,0.40,0.40}{##1}}}
\expandafter\def\csname PY@tok@mi\endcsname{\def\PY@tc##1{\textcolor[rgb]{0.40,0.40,0.40}{##1}}}
\expandafter\def\csname PY@tok@il\endcsname{\def\PY@tc##1{\textcolor[rgb]{0.40,0.40,0.40}{##1}}}
\expandafter\def\csname PY@tok@mo\endcsname{\def\PY@tc##1{\textcolor[rgb]{0.40,0.40,0.40}{##1}}}
\expandafter\def\csname PY@tok@ch\endcsname{\let\PY@it=\textit\def\PY@tc##1{\textcolor[rgb]{0.25,0.50,0.50}{##1}}}
\expandafter\def\csname PY@tok@cm\endcsname{\let\PY@it=\textit\def\PY@tc##1{\textcolor[rgb]{0.25,0.50,0.50}{##1}}}
\expandafter\def\csname PY@tok@cpf\endcsname{\let\PY@it=\textit\def\PY@tc##1{\textcolor[rgb]{0.25,0.50,0.50}{##1}}}
\expandafter\def\csname PY@tok@c1\endcsname{\let\PY@it=\textit\def\PY@tc##1{\textcolor[rgb]{0.25,0.50,0.50}{##1}}}
\expandafter\def\csname PY@tok@cs\endcsname{\let\PY@it=\textit\def\PY@tc##1{\textcolor[rgb]{0.25,0.50,0.50}{##1}}}

\def\PYZbs{\char`\\}
\def\PYZus{\char`\_}
\def\PYZob{\char`\{}
\def\PYZcb{\char`\}}
\def\PYZca{\char`\^}
\def\PYZam{\char`\&}
\def\PYZlt{\char`\<}
\def\PYZgt{\char`\>}
\def\PYZsh{\char`\#}
\def\PYZpc{\char`\%}
\def\PYZdl{\char`\$}
\def\PYZhy{\char`\-}
\def\PYZsq{\char`\'}
\def\PYZdq{\char`\"}
\def\PYZti{\char`\~}
% for compatibility with earlier versions
\def\PYZat{@}
\def\PYZlb{[}
\def\PYZrb{]}
\makeatother


    % Exact colors from NB
    \definecolor{incolor}{rgb}{0.0, 0.0, 0.5}
    \definecolor{outcolor}{rgb}{0.545, 0.0, 0.0}



    
    % Prevent overflowing lines due to hard-to-break entities
    \sloppy 
    % Setup hyperref package
    \hypersetup{
      breaklinks=true,  % so long urls are correctly broken across lines
      colorlinks=true,
      urlcolor=urlcolor,
      linkcolor=linkcolor,
      citecolor=citecolor,
      }
    % Slightly bigger margins than the latex defaults
    
    \geometry{verbose,tmargin=1in,bmargin=1in,lmargin=1in,rmargin=1in}
    
    

    \begin{document}
    
    
    \maketitle
    
    

    
    \section{Introduction}\label{introduction}

This is a simple example to explore the expected evolution of He-Pb
double dates.

Currently, I am going to explore a very simple scenario of a pluton that
cooled a relatively long time ago, and was then exhumed. This model is
also currently set up with the diffusion kinetics of Apatite, but that
needs to be updated to use Zircon (which is how this work is actually
done).

    \begin{Verbatim}[commandchars=\\\{\}]
{\color{incolor}In [{\color{incolor}1}]:} \PY{c+c1}{\PYZsh{}Import helpful functions}
        \PY{k+kn}{from} \PY{n+nn}{matplotlib} \PY{k}{import} \PY{n}{pyplot} \PY{k}{as} \PY{n}{plt} \PY{c+c1}{\PYZsh{}For plotting}
        \PY{k+kn}{from} \PY{n+nn}{matplotlib} \PY{k}{import} \PY{n}{cm} \PY{c+c1}{\PYZsh{}For colorbars}
        \PY{k+kn}{import} \PY{n+nn}{numpy} \PY{k}{as} \PY{n+nn}{np}
        \PY{o}{\PYZpc{}}\PY{k}{matplotlib} inline
        \PY{c+c1}{\PYZsh{}We want to import a local library too}
        \PY{k+kn}{import} \PY{n+nn}{sys}
        \PY{k+kn}{import} \PY{n+nn}{os}
        
        \PY{n}{sys}\PY{o}{.}\PY{n}{path}\PY{o}{.}\PY{n}{append}\PY{p}{(}\PY{n}{os}\PY{o}{.}\PY{n}{path}\PY{o}{.}\PY{n}{abspath}\PY{p}{(}\PY{l+s+s1}{\PYZsq{}}\PY{l+s+s1}{..}\PY{l+s+s1}{\PYZsq{}}\PY{p}{)}\PY{p}{)}
        \PY{k+kn}{import} \PY{n+nn}{Thermochronometer} \PY{k}{as} \PY{n+nn}{tchron} \PY{c+c1}{\PYZsh{}Import a local library that helps integrate thermochron ages}
        \PY{k+kn}{import} \PY{n+nn}{thermalHistory} \PY{k}{as} \PY{n+nn}{tHist}
\end{Verbatim}


    \section{Defining a cooling history}\label{defining-a-cooling-history}

The cooling history we are going to construct is as follows. A grain is
crystalized at some time as a pluton cools. The pluton quickly obtains a
geothermal gradient that is constant. At some point, exhumation begins
causing progressively deeper grains to be exhumed. These grains are
cooled at a constant rate that depends on the exhumation rate and the
geothermal gradient. Cooling of a grain continues until it reaches the
surface, and which point a small amount of time occurs while it is in
transport, and then the grain is deposited. Despite burial, the grain
maintains a constant temperature, which is that observed at Earth's
surface.

\section{Parameters}\label{parameters}

\begin{itemize}
\tightlist
\item
  \(U\) {[}L/t{]}: the exhumation rate
\item
  \(\frac{dT}{dz}\) {[}T/L{]}: A fixed geothermal gradients
\item
  \(T_s\) {[}T{]}: A surface temperature
\item
  \(Z_0\) {[}L{]}: the paleodepth or elevation from which a grain was
  eroded
\item
  \(t_u\) {[}t{]}: the time exhumation starts
\item
  \(t_{crys}\) {[}t{]}: the time that a grain was crystalized (e.g. U-Pb
  age)
\item
  \(t_d\) {[}t{]}: the time a grain was deposit (e.g. stratigraphic age)
\item
  \(\Delta t_l\) {[}t{]}: the time that a grain spent in transport
  (would probably start assuming this was 0)
\end{itemize}

    \begin{Verbatim}[commandchars=\\\{\}]
{\color{incolor}In [{\color{incolor}2}]:} \PY{c+c1}{\PYZsh{}Lets define some simple functions to create a thermal history that can be used to predict a double data plot}
        
        \PY{k}{def} \PY{n+nf}{verticalExhumationThermHistory}\PY{p}{(}\PY{n}{dTdZ}\PY{p}{,}\PY{n}{T\PYZus{}s}\PY{p}{,}\PY{n}{U}\PY{p}{,} \PY{n}{t\PYZus{}crys}\PY{p}{,}\PY{n}{t\PYZus{}u}\PY{p}{,}\PY{n}{t\PYZus{}d}\PY{p}{,}\PY{n}{delt\PYZus{}l}\PY{p}{,}\PY{n}{t}\PY{p}{)}\PY{p}{:}
            \PY{l+s+sd}{\PYZsq{}\PYZsq{}\PYZsq{}}
        \PY{l+s+sd}{    \PYZsq{}\PYZsq{}\PYZsq{}}
            
            \PY{n}{T} \PY{o}{=} \PY{n}{np}\PY{o}{.}\PY{n}{zeros\PYZus{}like}\PY{p}{(}\PY{n}{t}\PY{p}{)} \PY{c+c1}{\PYZsh{}Preallocate temperature output to match time axis}
            
            \PY{c+c1}{\PYZsh{}If we haven\PYZsq{}t started to exhume grains, everything is at the surface...}
            \PY{k}{if} \PY{n}{t\PYZus{}d} \PY{o}{\PYZgt{}} \PY{n}{t\PYZus{}u}\PY{p}{:}
                \PY{n}{T} \PY{o}{+}\PY{o}{=} \PY{n}{T\PYZus{}s}
            \PY{k}{else}\PY{p}{:}
        
                \PY{c+c1}{\PYZsh{}The grain is cooling once uplift begins, and stops cooling when it hits the surface}
                \PY{n}{isCooling} \PY{o}{=} \PY{p}{(}\PY{n}{t} \PY{o}{\PYZlt{}} \PY{n}{t\PYZus{}u}\PY{p}{)} \PY{o}{\PYZam{}} \PY{p}{(}\PY{n}{t} \PY{o}{\PYZgt{}}\PY{p}{(}\PY{n}{t\PYZus{}d} \PY{o}{+} \PY{n}{delt\PYZus{}l}\PY{p}{)}\PY{p}{)} \PY{c+c1}{\PYZsh{}Boolean array for the duration of cooling}
                \PY{n}{T}\PY{p}{[}\PY{n}{isCooling}\PY{p}{]} \PY{o}{=} \PY{n}{T\PYZus{}s} \PY{o}{+} \PY{n}{U}\PY{o}{*}\PY{p}{(}\PY{n}{t}\PY{p}{[}\PY{n}{isCooling}\PY{p}{]} \PY{o}{\PYZhy{}} \PY{p}{(}\PY{n}{t\PYZus{}d} \PY{o}{+} \PY{n}{delt\PYZus{}l}\PY{p}{)}\PY{p}{)}\PY{o}{*}\PY{n}{dTdZ}
        
                \PY{c+c1}{\PYZsh{}The grain is sitting at its original depth if cooling has yet to begin}
                \PY{n}{Z\PYZus{}0} \PY{o}{=} \PY{n}{U}\PY{o}{*}\PY{p}{(}\PY{n}{t\PYZus{}u} \PY{o}{\PYZhy{}} \PY{p}{(}\PY{n}{t\PYZus{}d}\PY{o}{+}\PY{n}{delt\PYZus{}l}\PY{p}{)}\PY{p}{)}
                \PY{n}{T}\PY{p}{[}\PY{n}{t}\PY{o}{\PYZgt{}}\PY{o}{=}\PY{n}{t\PYZus{}u}\PY{p}{]} \PY{o}{=} \PY{n}{T\PYZus{}s} \PY{o}{+} \PY{n}{Z\PYZus{}0}\PY{o}{*}\PY{n}{dTdZ}
        
                \PY{c+c1}{\PYZsh{}The grain is at the surface once t (the current time) is greater than the age of deposition minus the transport time}
                \PY{n}{T}\PY{p}{[}\PY{n}{t}\PY{o}{\PYZlt{}}\PY{o}{=}\PY{p}{(}\PY{n}{t\PYZus{}d} \PY{o}{+} \PY{n}{delt\PYZus{}l}\PY{p}{)}\PY{p}{]} \PY{o}{=} \PY{n}{T\PYZus{}s} 
            
            \PY{k}{return} \PY{n}{T}
            
\end{Verbatim}


    \begin{Verbatim}[commandchars=\\\{\}]
{\color{incolor}In [{\color{incolor}3}]:} \PY{c+c1}{\PYZsh{}Lets plot an example of that cooling history, first define some parameters}
        \PY{n}{U} \PY{o}{=} \PY{l+m+mf}{0.5} \PY{o}{/} \PY{l+m+mf}{1e3} \PY{c+c1}{\PYZsh{}mm/yr \PYZhy{}\PYZgt{} m/yr}
        \PY{n}{T\PYZus{}s} \PY{o}{=} \PY{l+m+mf}{5.0} \PY{c+c1}{\PYZsh{}Degrees in celsius}
        \PY{n}{dTdZ} \PY{o}{=} \PY{l+m+mf}{25.0} \PY{o}{/} \PY{l+m+mf}{1000.0} \PY{c+c1}{\PYZsh{}C/km \PYZhy{}\PYZgt{} C/m}
        \PY{n}{t\PYZus{}crys} \PY{o}{=} \PY{l+m+mf}{50.0} \PY{o}{*}\PY{l+m+mf}{1e6} \PY{c+c1}{\PYZsh{}Crystalization age}
        \PY{n}{delt\PYZus{}l} \PY{o}{=} \PY{l+m+mi}{0} \PY{c+c1}{\PYZsh{}Transport time of grain in years}
        \PY{n}{t\PYZus{}ds} \PY{o}{=} \PY{n}{np}\PY{o}{.}\PY{n}{array}\PY{p}{(}\PY{p}{[}\PY{l+m+mi}{20}\PY{p}{,}\PY{l+m+mi}{15}\PY{p}{,} \PY{l+m+mi}{10}\PY{p}{,} \PY{l+m+mi}{5}\PY{p}{,} \PY{l+m+mi}{1}\PY{p}{]}\PY{p}{)}\PY{o}{*}\PY{l+m+mf}{1e6} \PY{c+c1}{\PYZsh{}Ages of deposition of grains in Myr \PYZhy{}\PYZgt{} yrs}
        \PY{n}{t\PYZus{}u} \PY{o}{=} \PY{l+m+mf}{20.0} \PY{o}{*}\PY{l+m+mf}{1e6} \PY{c+c1}{\PYZsh{}Age of initiation of exhumation in Myr \PYZhy{}\PYZgt{} years}
        
        \PY{n}{dt\PYZus{}He} \PY{o}{=} \PY{l+m+mf}{50000.0} \PY{c+c1}{\PYZsh{}Resolution of cooling history}
        \PY{c+c1}{\PYZsh{}Create an axis for temperatures, sort it from young to old for giggles}
        \PY{n}{t} \PY{o}{=} \PY{n}{np}\PY{o}{.}\PY{n}{arange}\PY{p}{(}\PY{l+m+mi}{0}\PY{p}{,}\PY{n}{t\PYZus{}crys}\PY{p}{,}\PY{n}{dt\PYZus{}He}\PY{p}{)}\PY{p}{[}\PY{p}{:}\PY{p}{:}\PY{o}{\PYZhy{}}\PY{l+m+mi}{1}\PY{p}{]} 
        
        \PY{n}{cmap} \PY{o}{=} \PY{n}{cm}\PY{o}{.}\PY{n}{get\PYZus{}cmap}\PY{p}{(}\PY{l+s+s1}{\PYZsq{}}\PY{l+s+s1}{coolwarm\PYZus{}r}\PY{l+s+s1}{\PYZsq{}}\PY{p}{)}
        \PY{n}{colors} \PY{o}{=} \PY{p}{[}\PY{n}{cmap}\PY{p}{(}\PY{n}{i}\PY{p}{)} \PY{k}{for} \PY{n}{i} \PY{o+ow}{in} \PY{n}{np}\PY{o}{.}\PY{n}{linspace}\PY{p}{(}\PY{l+m+mi}{0}\PY{p}{,}\PY{l+m+mi}{1}\PY{p}{,}\PY{n+nb}{len}\PY{p}{(}\PY{n}{t\PYZus{}ds}\PY{p}{)}\PY{p}{)}\PY{p}{]} \PY{c+c1}{\PYZsh{}Create a colormap for lines}
        
        \PY{n}{f}\PY{p}{,}\PY{n}{axs} \PY{o}{=} \PY{n}{plt}\PY{o}{.}\PY{n}{subplots}\PY{p}{(}\PY{l+m+mi}{1}\PY{p}{,}\PY{l+m+mi}{1}\PY{p}{,}\PY{n}{figsize} \PY{o}{=} \PY{p}{(}\PY{l+m+mi}{10}\PY{p}{,}\PY{l+m+mi}{4}\PY{p}{)}\PY{p}{,}\PY{n}{dpi} \PY{o}{=} \PY{l+m+mi}{200}\PY{p}{)}
        
        \PY{c+c1}{\PYZsh{}Loop through the different depositional ages and make plots of those cooling histories}
        \PY{k}{for} \PY{n}{i}\PY{p}{,}\PY{n}{t\PYZus{}d} \PY{o+ow}{in} \PY{n+nb}{enumerate}\PY{p}{(}\PY{n}{t\PYZus{}ds}\PY{p}{)}\PY{p}{:}
            \PY{n}{T} \PY{o}{=} \PY{n}{verticalExhumationThermHistory}\PY{p}{(}\PY{n}{dTdZ}\PY{p}{,}\PY{n}{T\PYZus{}s}\PY{p}{,}\PY{n}{U}\PY{p}{,} \PY{n}{t\PYZus{}crys}\PY{p}{,}\PY{n}{t\PYZus{}u}\PY{p}{,}\PY{n}{t\PYZus{}d}\PY{p}{,}\PY{n}{delt\PYZus{}l}\PY{p}{,}\PY{n}{t}\PY{p}{)}
            
            \PY{n}{axs}\PY{o}{.}\PY{n}{plot}\PY{p}{(}\PY{n}{t}\PY{o}{/}\PY{l+m+mf}{1e6}\PY{p}{,}\PY{n}{T}\PY{p}{,}\PY{l+s+s1}{\PYZsq{}}\PY{l+s+s1}{\PYZhy{}}\PY{l+s+s1}{\PYZsq{}}\PY{p}{,}\PY{n}{color} \PY{o}{=} \PY{n}{colors}\PY{p}{[}\PY{n}{i}\PY{p}{]}\PY{p}{,}\PY{n}{label} \PY{o}{=} \PY{l+s+sa}{r}\PY{l+s+s1}{\PYZsq{}}\PY{l+s+s1}{\PYZdl{}t\PYZus{}d\PYZdl{} = }\PY{l+s+si}{\PYZob{}:.1f\PYZcb{}}\PY{l+s+s1}{ Ma}\PY{l+s+s1}{\PYZsq{}}\PY{o}{.}\PY{n}{format}\PY{p}{(}\PY{n}{t\PYZus{}d}\PY{o}{/}\PY{l+m+mf}{1e6}\PY{p}{)}\PY{p}{)}
        
        \PY{n}{axs}\PY{o}{.}\PY{n}{legend}\PY{p}{(}\PY{p}{)}
        \PY{n}{plt}\PY{o}{.}\PY{n}{xlabel}\PY{p}{(}\PY{l+s+s1}{\PYZsq{}}\PY{l+s+s1}{Time [Ma]}\PY{l+s+s1}{\PYZsq{}}\PY{p}{)}
        \PY{n}{plt}\PY{o}{.}\PY{n}{ylabel}\PY{p}{(}\PY{l+s+s1}{\PYZsq{}}\PY{l+s+s1}{Temperature [C]}\PY{l+s+s1}{\PYZsq{}}\PY{p}{)}
        \PY{n}{axs}\PY{o}{.}\PY{n}{invert\PYZus{}xaxis}\PY{p}{(}\PY{p}{)}
        \PY{n}{axs}\PY{o}{.}\PY{n}{invert\PYZus{}yaxis}\PY{p}{(}\PY{p}{)}
\end{Verbatim}


    \begin{center}
    \adjustimage{max size={0.9\linewidth}{0.9\paperheight}}{output_4_0.png}
    \end{center}
    { \hspace*{\fill} \\}
    
    \begin{Verbatim}[commandchars=\\\{\}]
{\color{incolor}In [{\color{incolor}4}]:} \PY{c+c1}{\PYZsh{}Now lets actually simulate some thermochronometers evolving from these thermal histories}
        \PY{c+c1}{\PYZsh{}Lets calculate this for more than just a few deposit ages}
        \PY{c+c1}{\PYZsh{}Evenly spaced depositional ages from 25\PYZpc{} more than start of uplift to modern}
        \PY{n}{t\PYZus{}ds} \PY{o}{=} \PY{n}{np}\PY{o}{.}\PY{n}{linspace}\PY{p}{(}\PY{n}{t\PYZus{}u}\PY{o}{*}\PY{l+m+mf}{1.25}\PY{p}{,}\PY{l+m+mi}{0}\PY{p}{,}\PY{l+m+mi}{20}\PY{p}{)} 
        
        
        \PY{c+c1}{\PYZsh{}First we need to define some things inside the thermochronometer}
        \PY{c+c1}{\PYZsh{}How big are grains?}
        \PY{n}{Radius} \PY{o}{=} \PY{l+m+mf}{100.0} \PY{o}{/} \PY{l+m+mf}{1e6}  \PY{c+c1}{\PYZsh{} Crystal radius in m}
        \PY{n}{dx} \PY{o}{=}\PY{l+m+mf}{1.0} \PY{o}{/} \PY{l+m+mf}{1e6}  \PY{c+c1}{\PYZsh{} spacing of nodes in m}
        \PY{n}{L} \PY{o}{=}  \PY{n}{np}\PY{o}{.}\PY{n}{arange}\PY{p}{(}\PY{n}{dx} \PY{o}{/} \PY{l+m+mf}{2.0}\PY{p}{,} \PY{n}{Radius}\PY{p}{,} \PY{n}{dx}\PY{p}{)}
        
        \PY{c+c1}{\PYZsh{} Concentrations of parent nuclides}
        \PY{n}{Conc238} \PY{o}{=} \PY{l+m+mf}{1.0}\PY{c+c1}{\PYZsh{}8.0}
        \PY{n}{Conc235} \PY{o}{=} \PY{p}{(}\PY{n}{Conc238} \PY{o}{/} \PY{l+m+mf}{137.0}\PY{p}{)}
        \PY{n}{Conc232} \PY{o}{=} \PY{l+m+mf}{1.0}\PY{c+c1}{\PYZsh{}147.0}
        
        \PY{c+c1}{\PYZsh{}Specify uniform concentration profiles for the parent isotopes (e.g., not zones)}
        \PY{n}{parentConcs} \PY{o}{=} \PY{n}{np}\PY{o}{.}\PY{n}{array}\PY{p}{(}\PY{p}{[}\PY{n}{Conc238} \PY{o}{*} \PY{n}{np}\PY{o}{.}\PY{n}{ones\PYZus{}like}\PY{p}{(}\PY{n}{L}\PY{p}{)}\PY{p}{,} 
                                \PY{n}{Conc235} \PY{o}{*} \PY{n}{np}\PY{o}{.}\PY{n}{ones\PYZus{}like}\PY{p}{(}\PY{n}{L}\PY{p}{)}\PY{p}{,} \PY{n}{Conc232} \PY{o}{*} \PY{n}{np}\PY{o}{.}\PY{n}{ones\PYZus{}like}\PY{p}{(}\PY{n}{L}\PY{p}{)}\PY{p}{]}\PY{p}{)}
        \PY{n}{daughterConcs} \PY{o}{=} \PY{n}{np}\PY{o}{.}\PY{n}{zeros\PYZus{}like}\PY{p}{(}\PY{n}{L}\PY{p}{)}
        
        \PY{c+c1}{\PYZsh{}What is the temperature\PYZhy{}dependent diffusivty we will use?}
        \PY{n}{diffusivity} \PY{o}{=} \PY{l+s+s1}{\PYZsq{}}\PY{l+s+s1}{Farley}\PY{l+s+s1}{\PYZsq{}}
        
        \PY{c+c1}{\PYZsh{}Create a model of a helium grain for each cooling history defined }
        \PY{c+c1}{\PYZsh{}by deposition at a different time}
        \PY{n}{HeModels} \PY{o}{=} \PY{p}{[}\PY{n}{tchron}\PY{o}{.}\PY{n}{SphericalApatiteHeThermochronometer}\PY{p}{(}\PY{n}{Radius}\PY{p}{,}
                            \PY{n}{dx}\PY{p}{,}\PY{n}{parentConcs}\PY{p}{,}\PY{n}{daughterConcs}\PY{p}{,}\PY{n}{diffusivityParams}\PY{o}{=}\PY{n}{diffusivity}\PY{p}{)}
                            \PY{k}{for} \PY{n}{t\PYZus{}d} \PY{o+ow}{in} \PY{n}{t\PYZus{}ds}\PY{p}{]}
        
        \PY{c+c1}{\PYZsh{}Thermal histories are a special function that can interpolate a discrete}
        \PY{c+c1}{\PYZsh{}set of time temperature points to predict diffusivity at any time}
        \PY{n}{thermalHistories} \PY{o}{=} \PY{p}{[}\PY{n}{tHist}\PY{o}{.}\PY{n}{thermalHistory}\PY{p}{(}\PY{o}{\PYZhy{}}\PY{n}{t}\PY{p}{,}\PY{l+m+mf}{273.15}\PY{o}{+}
                            \PY{n}{verticalExhumationThermHistory}\PY{p}{(}\PY{n}{dTdZ}\PY{p}{,}\PY{n}{T\PYZus{}s}\PY{p}{,}\PY{n}{U}\PY{p}{,} \PY{n}{t\PYZus{}crys}\PY{p}{,}\PY{n}{t\PYZus{}u}\PY{p}{,}\PY{n}{t\PYZus{}d}\PY{p}{,}\PY{n}{delt\PYZus{}l}\PY{p}{,}\PY{n}{t}\PY{p}{)}\PY{p}{)}
                            \PY{k}{for} \PY{n}{t\PYZus{}d} \PY{o+ow}{in} \PY{n}{t\PYZus{}ds}\PY{p}{]}
        
        \PY{c+c1}{\PYZsh{}Integrate these thermal histories}
        \PY{n}{Ages\PYZus{}He} \PY{o}{=} \PY{n}{np}\PY{o}{.}\PY{n}{zeros}\PY{p}{(}\PY{n+nb}{len}\PY{p}{(}\PY{n}{thermalHistories}\PY{p}{)}\PY{p}{)} \PY{c+c1}{\PYZsh{}}
        \PY{k}{for} \PY{n}{i}\PY{p}{,}\PY{n}{HeModel} \PY{o+ow}{in} \PY{n+nb}{enumerate}\PY{p}{(}\PY{n}{HeModels}\PY{p}{)}\PY{p}{:}
            \PY{c+c1}{\PYZsh{}Integrate this thermal history with a fixed timestemp}
            \PY{n}{HeModel}\PY{o}{.}\PY{n}{integrateThermalHistory}\PY{p}{(}\PY{o}{\PYZhy{}}\PY{n}{t}\PY{p}{[}\PY{l+m+mi}{0}\PY{p}{]}\PY{p}{,} \PY{l+m+mi}{0}\PY{p}{,}\PY{n}{dt\PYZus{}He}\PY{p}{,} \PY{n}{thermalHistories}\PY{p}{[}\PY{n}{i}\PY{p}{]}\PY{o}{.}\PY{n}{getTemp}\PY{p}{)}
            \PY{n}{Ages\PYZus{}He}\PY{p}{[}\PY{n}{i}\PY{p}{]} \PY{o}{=} \PY{n}{HeModel}\PY{o}{.}\PY{n}{calcAge}\PY{p}{(}\PY{n}{applyFt} \PY{o}{=} \PY{k+kc}{True}\PY{p}{)}
        
            
\end{Verbatim}


    \section{Summarizing the data}\label{summarizing-the-data}

One way to view changes in cooling ages upsection is by looking at the
'fractional helium loss' (\(f\), Reiners et al., 2005, Saylor et al.,
2012, Fosdick et al., 2015), defined as:

\[f = 100\% * \left(1 - \frac{Age_{He}}{Age_{U-Pb}}\right)\]

Values of 100\% indicate that daughter Helium has not been lost since
crystalization, and that the He age is reflecting the timing of
crystalisation defined by the U-Pb age (here \(t_{crys}\)). Values close
to zero indicate very recent and rapid cooling. Intermediate values
reflect loss of He as a result of the grains thermal history

    \begin{Verbatim}[commandchars=\\\{\}]
{\color{incolor}In [{\color{incolor}5}]:} \PY{c+c1}{\PYZsh{}Lets create a function to enable calculation of f}
        \PY{k}{def} \PY{n+nf}{fracLoss}\PY{p}{(}\PY{n}{Age\PYZus{}He}\PY{p}{,}\PY{n}{Age\PYZus{}UPb}\PY{p}{)}\PY{p}{:}
            \PY{k}{return} \PY{l+m+mf}{100.0}\PY{o}{*} \PY{p}{(}\PY{l+m+mf}{1.0} \PY{o}{\PYZhy{}} \PY{n}{Age\PYZus{}He}\PY{o}{/}\PY{n}{Age\PYZus{}UPb}\PY{p}{)}
        
        \PY{c+c1}{\PYZsh{}Calculate F values for each point in the \PYZsq{}stratigraphic section\PYZsq{} of modelled He ages}
        \PY{n}{fs} \PY{o}{=} \PY{n}{np}\PY{o}{.}\PY{n}{zeros\PYZus{}like}\PY{p}{(}\PY{n}{t\PYZus{}ds}\PY{p}{)}
        
        \PY{k}{for} \PY{n}{i} \PY{o+ow}{in} \PY{n+nb}{range}\PY{p}{(}\PY{n+nb}{len}\PY{p}{(}\PY{n}{fs}\PY{p}{)}\PY{p}{)}\PY{p}{:}
            \PY{n}{fs}\PY{p}{[}\PY{n}{i}\PY{p}{]} \PY{o}{=} \PY{n}{fracLoss}\PY{p}{(}\PY{n}{Ages\PYZus{}He}\PY{p}{[}\PY{n}{i}\PY{p}{]}\PY{p}{,}\PY{n}{t\PYZus{}crys}\PY{o}{/}\PY{l+m+mf}{1e6}\PY{p}{)}
            
        \PY{c+c1}{\PYZsh{}Plot the results}
        \PY{n}{f}\PY{p}{,}\PY{n}{axs} \PY{o}{=} \PY{n}{plt}\PY{o}{.}\PY{n}{subplots}\PY{p}{(}\PY{l+m+mi}{1}\PY{p}{,}\PY{l+m+mi}{1}\PY{p}{,}\PY{n}{figsize} \PY{o}{=} \PY{p}{(}\PY{l+m+mf}{2.5}\PY{p}{,}\PY{l+m+mi}{6}\PY{p}{)}\PY{p}{,}\PY{n}{dpi} \PY{o}{=} \PY{l+m+mi}{200}\PY{p}{)}
        \PY{n}{axs}\PY{o}{.}\PY{n}{plot}\PY{p}{(}\PY{n}{fs}\PY{p}{,}\PY{n}{t\PYZus{}ds}\PY{o}{/}\PY{l+m+mf}{1e6}\PY{p}{,}\PY{l+s+s1}{\PYZsq{}}\PY{l+s+s1}{\PYZhy{}k}\PY{l+s+s1}{\PYZsq{}}\PY{p}{,}\PY{n}{label} \PY{o}{=} \PY{l+s+s1}{\PYZsq{}}\PY{l+s+s1}{Prediction}\PY{l+s+s1}{\PYZsq{}}\PY{p}{)}
        \PY{n}{axs}\PY{o}{.}\PY{n}{invert\PYZus{}yaxis}\PY{p}{(}\PY{p}{)}
        \PY{n}{plt}\PY{o}{.}\PY{n}{xlabel}\PY{p}{(}\PY{l+s+s1}{\PYZsq{}}\PY{l+s+s1}{Fractional Loss}\PY{l+s+s1}{\PYZsq{}}\PY{p}{)}
        \PY{n}{plt}\PY{o}{.}\PY{n}{ylabel}\PY{p}{(}\PY{l+s+s1}{\PYZsq{}}\PY{l+s+s1}{Depositional Age}\PY{l+s+s1}{\PYZsq{}}\PY{p}{)}
        
        \PY{c+c1}{\PYZsh{}Plot some reference lines}
        \PY{n}{axs}\PY{o}{.}\PY{n}{plot}\PY{p}{(}\PY{p}{[}\PY{l+m+mi}{0}\PY{p}{,}\PY{l+m+mi}{100}\PY{p}{]}\PY{p}{,}\PY{p}{[}\PY{n}{t\PYZus{}u}\PY{o}{/}\PY{l+m+mf}{1e6}\PY{p}{,}\PY{n}{t\PYZus{}u}\PY{o}{/}\PY{l+m+mf}{1e6}\PY{p}{]}\PY{p}{,}\PY{l+s+s1}{\PYZsq{}}\PY{l+s+s1}{\PYZhy{}\PYZhy{}k}\PY{l+s+s1}{\PYZsq{}}\PY{p}{,}\PY{n}{label} \PY{o}{=} \PY{l+s+s1}{\PYZsq{}}\PY{l+s+s1}{Initation of exhumation}\PY{l+s+s1}{\PYZsq{}}\PY{p}{)}
        
        \PY{n}{axs}\PY{o}{.}\PY{n}{legend}\PY{p}{(}\PY{p}{)}
\end{Verbatim}


\begin{Verbatim}[commandchars=\\\{\}]
{\color{outcolor}Out[{\color{outcolor}5}]:} <matplotlib.legend.Legend at 0x1821af1898>
\end{Verbatim}
            
    \begin{center}
    \adjustimage{max size={0.9\linewidth}{0.9\paperheight}}{output_7_1.png}
    \end{center}
    { \hspace*{\fill} \\}
    

    % Add a bibliography block to the postdoc
    
    
    
    \end{document}
